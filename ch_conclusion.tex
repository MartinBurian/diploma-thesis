\documentclass[buriama8_dp.tex]{subfiles}
\begin{document}

\chapter{Conclusions}

The taks of exploring space with a robotic arm proved to be very complex. From the hardware and the low-level concepts of serial manipulator, through robotic software frameworks with not-so-well functioning libraries, to covering unknown spaces, all levels are critical to complete the task.


\section{Results of our work}
\label{sec:label}

We designed and buil a tool that extends a force sensor sensing area further away from the arm, while maintaining its sensitivity. The tool is made to be held in the arm's gripper, and can thus be attached to the arm without any tools, even by the robot itself. We also modified the device driver to provide the data we needed. The sensor data are calibrated to the scale in Newtons.

The JACO arm was previously used in the TRADR project only in conjunction with motion planning. We developed a control module that can drive the arm along straight trajectories with reasonable precision of about 2\,cm. This mode of control allowed us to speed the whole exploration process up by by-passing the previously necessary, time consuming and computationally expensive motion planning. The controller cannot cope with everything and motion planning is still invaluable for longer motions and motions around arm configurations where the movement cannot be controlled analytically, and over the course of our work, many issues with the planners were identified. Some have been solved, other remain.

We created and tested a Coverage Path Planning algorithm for unknown environemtns, the compact space heuristic algorithm, that guides the arm exploration. Although the algorithm is heuristic in nature, and as such does not give any optimality guarantees, the exploration trajectories it generates are both short and robot friendly. The compactness of explored space maximizes the robot's operation space, simplifying the planning problems. It is also well-suited to be used in conjunction with grid maps and the simple linear motion control.

All the previously mentioned things were stitched together to create a prototype of the system that could be put to the test. Only wrapping the functionality into an API is needed to make the exploration usable by other system components.

The performance of the complete solution was verified in both simulation and \TODO{dopiš to}

\section{Further work proposals}
\label{sec:label}

Some components in the system, namely the \code{FollowJointTrajectoryAction} server, are implemented in such a way that they work, but have issues.

\begin{enumerate}
\item improve FollowJointTrajectory
\item improve joint limits
\end{enumerate}



\end{document}

%%% Local Variables:
%%% mode: latex
%%% TeX-master: "buriama8_dp"
%%% End:
