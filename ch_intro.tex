
\documentclass[buriama8_dp.tex]{subfiles}
\begin{document}

\chapter{Introduction}

\section{Introduction}


\section{Exploration task overview}
\label{sec:label}

Our task is to use a robotic arm to explore space in front of the robot to identify obstacles that could damage it. In the big picture, we need to build a piece of software that will drive the robotic arm hardware around the environment intelligently, not crashing the arm into anything in the process, and via this movement explore the environment in front of the robot. All the while, it will be processing data from a 3D force sensing tool we designed to detect the obstacles. To cover the whole complexity of driving a robotic arm in an environment so that all the environment is explored, we will need to acquaint the reader with the workings of many a field, from robotic manipulators to planning efficient exploring paths.

The whole system (the robotic platform and the arm) run in the Robot Operating System (\emph{ROS}) framework. To maintain system consistency and simplify implementation, we will be implementing our solution in ROS as well. We provide an overview of the framework in Section~\ref{sec:ros}, where we briefly present the basic concepts to help the reader grasp what is underlying our implementation decisions.

The arm is an advanced piece of hardware and embedded programming, with builtin controls capable of driving the arm in space. The functionality is exposed to the system via a manufacturer-provided driver. We describe both the arm hardware and driver in Section~\ref{sec:jaco}. We also describe the force sensor and its extension we designed for this application. 

Although the driver exposes simple interfaces to control the arm movement, it is neccessary to control the arm's motion in context of the whole environment including the robotic platform and any obstacles around the robot. This task is solved by \uvz{MoveIt!}, a library built around functionalities required to use robotic manipulators. We present the library in Section~\ref{sec:moveit}. We also include information about the mathematical and algorithmic framework of robotic manipulators and planning and controlling their motion that is crucial for solving our task.

\section{Outline of the thesis}
\label{sec:outline}

First, we will present the hardware used in the exploration task, the Kinova JACO robotic arm and the Optoforce 3D force sensor with our custom extension. We also present the software framework 

Next, we present the overview of state-of-the-art methods of coverage path planning, and the rationale behind our method choice.

Later, we describe the implementation of the algorithm in ROS and experiments with the exploration methods, both simulated and on a real robotic arm.



\end{document}
%%% Local Variables:
%%% mode: latex
%%% TeX-master: "buriama8_dp"
%%% End:
